\section*{Постановка задачи}
Необходимо сделать нормальный шаблон для отчётов в Политехе. Структура отчётов
может быть разной, зависит от требования преподавателя, поэтому файл
content.tex отдельно выделен от всех других в шаблоне и не делится на подчасти.
\addcontentsline{toc}{section}{Постановка задачи}

\newpage
\section{Заполнение шаблона}
\begin{itemize}
  \item Изменить \textbf{config.tex}: имя студента, название предмета и пр.
    параметры указаны именно там
  \item Заполнить \textbf{content.tex} - файл, который будет содержать весь
    текст отчёта, от вступления до заключения.
  \item Добавить используемую литературу (если есть) в \textbf{refs.bib}. Для
    удобного поиска источников можно воспользоваться Google Books.
    Использованные источники можно указывать с помощью команды
    \textbf{\\cite\{name\_of\_ref\}}
\end{itemize}
Далее представлены различные примеры.

\section{Теоретическая информация}
bash \cite{bash} \\

\section{Ход выполнения работы}

\subsection{Список}

\begin{itemize}
	\item первый элемент списка
	\item второй элемент списка
\end{itemize}


\subsection{Картинка}

\begin{figure}[H]
	\begin{center}
		\includegraphics[scale=0.7]{sample}
		\caption{название картинки}
		\label{pic:pic_name} % название для ссылок внутри кода
	\end{center}
\end{figure}

Текст без отступа (следует за вставкой)

Новый параграф

\noindent Новый параграф с принудительно выключенным отступом


\subsection{Таблицы}

\begin{table}[H]
	\caption{Одна таблица}
	\begin{center}
		\begin{tabular*}{0.4\textwidth}{@{\extracolsep{\fill} } lcc}
			\toprule
			Element & First & Second \\
			\midrule
			One       & -    & -    \\
			Two       & -    & -    \\
			Three     & -    & -    \\
			Four      & -    & -    \\
			\bottomrule
		\end{tabular*}
		\label{tabular:tab_examp_1}
	\end{center}

	\caption{Другая таблица}
	\begin{center}
		\begin{tabular}{|l|c|r|}
			\hline
			top left & top center & top right \\ \hline
			bot left & bot center & bot right \\ \hline
		\end{tabular}
		\label{tabular:tab_examp_2}
	\end{center}
\end{table}

\subsection{Листинг}
\begin{code}
	\lstinputlisting[language=C]{listings/main.c}
	\caption{main.c – процедурный код в массы!}
\end{code}

\newpage
\section*{Заключение}
\LaTeX\ удобен для создания отчётов, так как сам следит за нумерацией таблиц,
рисунков, листингов и отсылок к ним (так, например, здесь всегда будет указан
номер рисунка "sample" не зависимо от того, какой он (1,2 или другой) - это
рисунок \ref{pic:pic_name}). Не менее важно что весь документ оформлен в едином
стиле, а исходные материалы подключаются к отчёту, а не хранятся в нём. Всё это
позволяет легко получить качественный отчёт без дополнительных трат на его
офрмление.

Исключения, пожалуй, составляют таблицы, так как их значительно сложнее
создавать кодом, нежели в графическом редакторе. Но здесь никто не запрещает
использовать визуальные средства создания таблиц для \LaTeX\ .
\addcontentsline{toc}{section}{Заключение}
